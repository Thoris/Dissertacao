O objetivo deste projeto de mestrado é desenvolver técnicas de controle subótimo das juntas passivas (não atuadas) de um robô subatuado, incluindo o estudo teórico do tema, proposição de um método de controle e sua verificação
experimental em um manipulador de três graus de liberdade \cite{Nascimento1970}.

O teste \cite{Patagonios2001} e validação das técnicas de controle propostas foram realizados em um ambiente de simulação e no manipulador
experimental, adquirido através do projeto FAPESP $N^{\circ}$ 98/00649-5, que se encontra em funcionamento no Laboratório de Sistemas Inteligentes (LASI) do Departamento de Engenharia Elétrica da USP em São Carlos. Pode-se citar \cite{Furmento1995}:
\begin{itemize}
\item Isso;
\item Aquilo; e
\item Aquele outro.
\end{itemize}